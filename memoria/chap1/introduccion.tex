\chapter{Propuesta}

\section{Descripción general de la aplicación}

Se desarrollará un programa interactivo que brinde a los usuarios la capacidad
de aprender a resolver el Cubo de Rubik paso a paso. El programa proporcionará
una interfaz intuitiva y amigable consciente del contexto que permitirá a los
usuarios interactuar con el cubo reconociendo su posición y proporcionando guías
visuales-auditivas para realizar los movimientos necesarios.

\subsection{Papel que juegan las tecnologías involucradas}

\begin{itemize}
\item{\textbf{Reconocimiento e Identificación de Colores}: Basándonos el proyecto QBR \cite{QBR},
utilizaremos técnicas de reconocimiento de imágenes para detectar y reconocer
los diferentes colores de las caras del cubo de Rubik. El programa será capaz de
interpretar la posición actual del cubo en función de la disposición de los
colores detectados.}

\item{ \textbf{Realidad Aumentada}: Utilizaremos la tecnología de realidad aumentada para
superponer un cubo 3D en tiempo real mediante un marcador ARUCO. La realidad
aumentada proporcionará una experiencia inmersiva y facilitará la comprensión de
los movimientos necesarios para resolver el cubo.}

\item{ \textbf{Procesamiento de Lenguaje Natural}: Implementaremos procesamiento de lenguaje
natural para brindar instrucciones claras y comprensibles al usuario. El
programa podrá generar instrucciones paso a paso, describir los movimientos
necesarios y proporcionar consejos para resolver el cubo de Rubik.}


\item{ \textbf{Consciencia del Contexto}: El programa será consciente del usuario con el que
interactúa mediante reconocimiento facial lo que permitirá un tratamiento
personalizado y una identificación con respecto a sus preferencias.}

\end{itemize}